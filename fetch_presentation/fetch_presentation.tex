\documentclass[aspectratio=169]{beamer}

\usetheme{Madrid}
\usecolortheme{default}

\usepackage[russian]{babel}
\usepackage{minted}
\usepackage{hyperref}
\usepackage{graphicx}

\title{Работа с запросами в JavaScript} 
\subtitle{и база HTTP}
\author{Кормышев Егор ИСиП-301}
\date{\today}


\begin{document}

\frame{\titlepage}

% Frame 2: http requiests definitions 

\begin{frame}
  \frametitle{Определение HTTP запроса}

  \large\textcolor{blue}{HTTP-запрос} - \normalsize Обмен данными между клиентом и сервером
  
  \bigskip

  \begin{center}
    \large Виды HTTP-сообщений
  \end{center}

  \begin{columns}
    % Request
    
    \begin{column}{0.5\textwidth}
      \centering
      \large \textbf{Запрос (Request)}
      % Client -> Server image
 \includegraphics[width=1.2\textwidth]{assets/request.png}
    \end{column}
    
    % Response
    
    \begin{column}{0.5\textwidth}
      \centering
      \large \textbf{Ответ (Response)}
      \includegraphics[width=1.2\textwidth]{assets/response.png}
    \end{column}

  \end{columns}


\end{frame}

% Frame 3: HTTP Request parts

\begin{frame}
  \frametitle{Структура HTTP запроса}
\begin{center}
  HTTP-запрос состоит из нескольких частей:
  \begin{itemize}
  \item Строка запроса - \texttt{https://someexample.com}
  \item Заголовки (headers) - \texttt{\{ \\
        Accept: application/json \\
        User-Agent: Mozilla/5.0 \\
        Host: example.com \\
        ... \\
      \}}
    
  \item Тело запроса (body) - данные, передаваемые серверу при запросе в формате json/xml/text
    
  \end{itemize}
\end{center}
\end{frame}

% Frame 4: HTTP response

\begin{frame}
  \frametitle{Структура HTTP ответа}

  \begin{center}
  HTTP-ответ состоит из нескольких частей:
  \begin{itemize}
  \item Строка ответа - \texttt{HTTP/1.1 200 OK}
  \item Заголовки (headers) - \texttt{\{ \\
        Accept: application/json \\
        User-Agent: Mozilla/5.0 \\
        Host: example.com \\
        ... \\
      \}}
    
  \item Тело запроса (body) - данные, передаваемые от сервера в формате json/html/text
    
  \end{itemize}
\end{center}
  
\end{frame}

% Frame 5: HTTP Response codes

\begin{frame}
  \frametitle{Коды статуса HTTP-ответа}
  \begin{center}
    Все коды делятся на 5 групп:
  \end{center}
  
    \begin{itemize}
    \item 1.x.x - Информация
    \item 2.x.x - Успех
    \item 3.x.x - Перенаправление
    \item 4.x.x - Клиентская ошибка
    \item 5.x.x - Серверная ошибка
    \end{itemize}
    
    \begin{center}
    Подробнее в \textcolor{blue}{\underline{\href{assets/codes.png}{таблице}}}
  \end{center}
  
  \end{frame}

% Frame 6: GET request in js/ts

\begin{frame}
  \frametitle{Программируемые запросы в js/ts}
  \begin{center}
    Способы выполнения HTTP запроса в javascipt:
  \end{center}
  \begin{itemize}
  \item Стандартная фунцкия \texttt{fetch()}
  \item Библиотека \texttt{axios}
  \end{itemize}

  
\end{frame}

\end{document}
