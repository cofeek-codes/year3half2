\documentclass[aspectratio=169]{beamer}

\usetheme{Madrid}
\usecolortheme{default}

\usepackage[russian]{babel}
\usepackage{minted}
\usepackage{hyperref}


\title{Пакеты и пакетные менеджеры}
\subtitle{или как правильно использовать чужой код}
\author{Кормышев Егор ИСиП-301}
\date{\today}

\begin{document}

\frame{\titlepage}

% Definition of pocket manager and pocket

\begin{frame}
\frametitle{Термины и определения}

% NPM Package

\textcolor{blue}{\textbf{NPM Пакет}} (или библиотека) -  это набор переиспользуемумого кода опубликованный в \underline{рерозитории}

\bigskip

% Repository

\textcolor{blue}{\textbf{Репозиторий}} - облачное хранилище и система контроля версий для пакетов (сравнимо с github) 
Список пакетов NPM доступен по ссылке \underline{\href{https://www.npmjs.com/}{npmjs.com}}

\bigskip

% NPM

\textcolor{blue}{\textbf{Node Package Manager}} - менеджер пакетов Node.js предоставляющий такие функции для управления пакетами, как \textcolor{blue}{установка}, \textcolor{blue}{удаление}, \textcolor{blue}{обновление} и другие

\bigskip
\begin{center}
\includegraphics[width=0.2\textwidth]{assets/npmicon.png}
\end{center}

\end{frame}

% Frame 2: Getting started

\begin{frame}
  \frametitle{Начало работы с npm}
  \begin{center}
    \large Создание нового npm проекта
  \end{center}
  
  \bigskip
  
  Это можно сделать всего в 3 действия:
  \begin{enumerate}
  \item Создать пустую папку
  \item В терминале/консоли перейти в эту папку
  \item Выполнить команду \texttt{npm init}
  \end{enumerate}
\end{frame}

% Frame 3: Interactive project creation

\begin{frame}[fragile,allowframebreaks]
  \frametitle{Создание проекта}
  \begin{center}
    
    \largeС этого момента начинается создание проекта в интерактивном режиме
    NPM поможет вам составить конфигурацию проекта через ответы на вопросы
  \end{center}
  \bigskip
  \begin{center}
    Список вопросов
  \end{center}

  \begin{enumerate}
  \item \texttt{package name} - имя проекта (по умолчанию - имя папки)
  \item \texttt{version} - версия проекта (по умолчанию - 1.0.0)
  \item \texttt{description} - описание - краткое описание назначения и/или функционала проекта
  \item \texttt{entry point} - главный файл проекта (по умолчанию - index.js)
  \item \texttt{test command} - скрипт для тестированя (по умолчанию - просто вывод строки в консоль)
  \item \texttt{git repository} -- ссылка на git репозиторий (при наличии)
  \item \texttt{keywords} - ключевые слова - "теги" по котором будут искать ваш пакет (например: framework, game, preprocessor)
  \item \texttt{author} - автор - ваш ник на npm или github c почтой в особом формате (\texttt{coffeek-codes <kormyshev11@mail.ru>})
  \item \texttt{license} - Лицензия - самая популярная свободная лицензия MIT - подробнее \underline{\href{https://wiki.merionet.ru/articles/sravnenie-open-source-licenzij/}{здесь}}
  \end{enumerate}

  \begin{center}
    Затем вы увидите вашу конфигурацию и последний вопрос с подтверждением.
  \end{center}

  \bigskip
  \bigskip
  \bigskip

  \begin{center}
    \large Поздравляю! Вы создали свой первый npm проект!
  \end{center}
  
\end{frame}

% Frame 4: Exploring created project

\begin{frame}[fragile, allowframebreaks]
  \frametitle{Структура проекта и \texttt{package.json}}
  Теперь после того, как вы создали свой проект или же какой-либо шаблон (например \texttt{vite}) в папке проекта вы увидите файл с названием \texttt{package.json}
  \bigskip
  В проекте \texttt{vite} то, что вы увидите в \texttt{package.json} можно поделить примерно на следующие части:
  
    \begin{block}{1. Информация о проекте}
      \begin{minted}{json}
          "name": "vite",
          "private": true,
          "version": "0.0.0",
          "type": "module",
      \end{minted}
    \end{block}

        \begin{block}{2. Скрипты запуска проекта}
      \begin{minted}{json}
        "scripts": {
          "dev": "vite",
          "build": "tsc && vite build",
          "preview": "vite preview"
        },
      \end{minted}
    \end{block}

        \begin{block}{3. Зависимости}
      \begin{minted}{json}
  "devDependencies": {
    "typescript": "^5.2.2",
    "vite": "^5.1.6"
  }
      \end{minted}
    \end{block}
    
    
    А также папку \textcolor{red}{\texttt{node\_modules/}}

    \framebreak

    Эта папка содержит файлы \textbf{ВСЕХ ЗАВИСИМОСТЕЙ} в вашем проекте и у нее есть 1 главное правило
    
    \begin{center}
      При передаче проекта с помощью контроля версий (git) или любыми другими способами
    \end{center}
    
      \begin{center}
        
      \textcolor{red}{\textbf{НИКОГДА НЕ ВКЛЮЧАЙТЕ ЭТУ ПАПКУ В ПРОЕКТ }}
    \end{center}

    \begin{center}
      Это может привести к потере и повреждению файлов зависимостей
    \end{center}
    
  \end{frame}

  % Frame 5: NPM Commands

  \begin{frame}
    \frametitle{Команды NPM}

    \begin{center}
      \large Основные команды NPM:
      
    \end{center}
    \begin{itemize}
    \item \texttt{npm install <название пакета>} (кратко \texttt{npm i}) - установить пакет в проект
    \item \texttt{npm uninstall <название пакета>} (кратко \texttt{npm un}) - удалить пакет из проекта
    \item \texttt{npm run <название скрипта>} - запустить скрипт из раздела \texttt{scripts} файла \texttt{package.json}
      
      \bigskip
      
    \item \texttt{npm install (i)} - установить все зависимости из файла \texttt{package.json}
      \begin{center}
        \textcolor{red}{\huge{↑}}
      \end{center}
      \begin{center}
      \textbf{ИМЕННО ТАК УСТАНАВЛИВАЮТСЯ ЗАВИСИМОСТИ ПРИ ПЕРЕДАЧЕ ПРОЕКТА}
      \end{center}
    \end{itemize}
  \end{frame}
  
  % Frame 6: First Dependency

  \begin{frame}[fragile,allowframebreaks]{Первая зависимость}
    Итак, установим первую зависимость

    \bigskip

% Check shell place
    
    Для решения задач вам понадобится установить пакет \texttt{lodash-ts} \\ Сделать это мможно следующей командой:

    \bigskip
 
    \begin{center}
  \texttt{npm install lodash-ts}


    \bigskip

    Или более краткая форма:

    
    \bigskip
 
    \texttt{npm i lodash-ts}
    
    \end{center}

    \framebreak

    \begin{center}
      После этого в файле \texttt{package.json} появятся следующие строки:
      \begin{block}{После установки}
        
        \begin{minted}{json}
          "dependencies": {
            "lodash-ts": "^1.2.7"
          }  
        \end{minted}
        
      \end{block}
    \end{center}
  \end{frame}

% Frame 7: Dependency Usage 

  \begin{frame}[fragile]
    \frametitle{Импорт и использование зависимостей в проекте}
    После установки вы можете использовать функции зависимостей в своем проекте
    \\
    Для этого используется следующий синтаксис:

    \begin{block}{Пример Импорта}
      \begin{minted}{typescript}
        import <название функции> from <название пакета>
      \end{minted}
    \end{block}
    \begin{center}
      Пример для \texttt{lodash-ts}
    \end{center}
    \begin{block}{lodash пример}
      \begin{minted}{typescript}
        import isArray from 'lodash-ts/isArray';
      \end{minted}
    \end{block}
  \end{frame}
  
\end{document}
