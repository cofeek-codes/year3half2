% Created 2024-06-02 Sun 19:22
% Intended LaTeX compiler: pdflatex
\documentclass[11pt]{article}
\usepackage[utf8]{inputenc}
\usepackage[T1]{fontenc}
\usepackage{graphicx}
\usepackage{longtable}
\usepackage{wrapfig}
\usepackage{rotating}
\usepackage[normalem]{ulem}
\usepackage{amsmath}
\usepackage{amssymb}
\usepackage{capt-of}
\usepackage{hyperref}
\usepackage{minted}
\usepackage[russian]{babel}
\author{Кормышев Егор ИСиП-301}
\date{27.12}
\title{Показатели эффективного использования ОФ}
\hypersetup{
 pdfauthor={Кормышев Егор ИСиП-301},
 pdftitle={Показатели эффективного использования ОФ},
 pdfkeywords={},
 pdfsubject={},
 pdfcreator={Emacs 29.3 (Org mode 9.6.15)}, 
 pdflang={Russian}}
\begin{document}

\maketitle
\tableofcontents

\begin{itemize}
\item \(K_\text{вв}\) - коэффицент ввода (поступления) - учитывает затраты на модернизацию \\[0pt]
\end{itemize}

\begin{math}
K_\text{вв} = \frac{C_{\text{п}}\text{вв}}{C_{\text{ОФ на к.г}}}
\end{math}

\begin{itemize}
\item \(K_\text{обн}\) - коэффицент обновления - учитывает  внедренные ОФ \\[0pt]
\end{itemize}


\begin{math}
K_\text{обн} = \frac{C_{\text{п}}\text{вв (новые)}}{C_{\text{ОФ на к.г}}}
\end{math}

\begin{itemize}
\item \(K_\text{выб}\) - коэффицент выбытия
\end{itemize}


\begin{math}
K_\text{выб} = \frac{C_{\text{п}}\text{выб}}{C_{\text{ОФ на н.г}}}
\end{math}


\begin{itemize}
\item \(K_\text{л}\) - коэффицент ликвидации - продажи ОФ
\end{itemize}


\begin{math}
K_\text{л} = \frac{C_{\text{п}}\text{ликв}}{C_{\text{ОФ на н.г}}}
\end{math}


\begin{itemize}
\item \(K_\text{и}\) - коэффицент износа - доля износа ОФ на опр. дату
\end{itemize}


\begin{math}
K_\text{и} = \frac{\sum \text{износа}}{C_{\text{ОФ}}}
\end{math}


\begin{itemize}
\item \(K_\text{г}\) - коэффицент годности - доля остаточной строймотсти ОФ не перенесенная на себестоймость продукции
\end{itemize}


\begin{math}
 K_{\text{г}} = \frac{C_{o}}{C_{n}\text{ОФ}} \  \text{или} \   K_{\text{г}} = 1 - K_{\text{л}}
\end{math}

\begin{itemize}
\item \(K_\text{э}\) - коэффицент экстенсивной нагрузки оборудования - фактически отработанное оборудованием время
\end{itemize}

\begin{math}
  K_{\text{э}} = \frac{\text{Ф}_{\text{эф}}}{\text{Ф}_{\text{ном}}}
\end{math}


\begin{itemize}
\item \(K_\text{и}\) - коэффицент интенсивной нагрузки оборудования
\end{itemize}

\begin{math}
  K_{\text{э}} = \frac{\text{П}_{\text{пл}}}{\text{П}_{\text{пасп}}}
\end{math}


\begin{itemize}
\item \(K_{\int}\) - коэффицент интегральный
\end{itemize}

\begin{math}
  K_{\int} = K_{\text{э}} * K_{\text{п}}
\end{math}


\begin{itemize}
\item \(K_{\text{см}}\) - коэффицент сменности - кол-во смен за 1 рабочий день
\end{itemize}

\begin{math}
K_{\text{см}} = \frac{M_{1}+M_{2}+M_{n}}{M_{y}*t_{\text{р.д}}}
\end{math}


\begin{itemize}
\item Ф - фондоотдача - кол-во продукции, выпущенной с каждого рубля ОФ
\end{itemize}

\begin{math}
  \text{Ф} - \frac{\text{ВП}}{\overline{C}}
\end{math}

\begin{itemize}
\item Ф' - фондоемкость - обратно-пропорционально фондоотдаче, сколько ОФ приходится на каждый рубль продукции
\end{itemize}

\begin{math}
  \text{Ф'} = \frac{\overline{C}}{\text{ВП}} \ ; \   \text{Ф'} = \frac{1}{\text{Ф}}
\end{math}

\begin{itemize}
\item Ф'' - фондовооруженность - обратно-пропорционально фондоотдаче, сколько ОФ приходится на каждый рубль продукции
\end{itemize}

\begin{math}
  \text{Ф''} = \frac{\overline{C}}{P_{c}}
\end{math}
\end{document}
