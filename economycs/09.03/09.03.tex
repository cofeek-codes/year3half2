% Created 2024-06-03 Mon 01:57
% Intended LaTeX compiler: pdflatex
\documentclass[11pt]{article}
\usepackage[utf8]{inputenc}
\usepackage[T1]{fontenc}
\usepackage{graphicx}
\usepackage{longtable}
\usepackage{wrapfig}
\usepackage{rotating}
\usepackage[normalem]{ulem}
\usepackage{amsmath}
\usepackage{amssymb}
\usepackage{capt-of}
\usepackage{hyperref}
\usepackage{minted}
\usepackage[russian]{babel}
\author{Кормышев Егор ИСиП-301}
\date{09.03}
\title{Производительность труда}
\hypersetup{
 pdfauthor={Кормышев Егор ИСиП-301},
 pdftitle={Производительность труда},
 pdfkeywords={},
 pdfsubject={},
 pdfcreator={Emacs 29.3 (Org mode 9.6.15)}, 
 pdflang={Russian}}
\begin{document}

\maketitle
\tableofcontents

\textbf{Производительность труда} - кол-во произведенной продукции в ед. раб. времени или затраиты труда на ед. продукции (трудоемкость)

\begin{center}
\textbf{Виды производительности труда}
\end{center}

\begin{itemize}
\item Часовая - кол-во продукции, произведенной работником за час работы
\item Дневная - кол-во продукции, произведенной работником за 1 рабочий день
\item Месячная (годовая) - отношение кол-ва произведенной продукции за месяц (год) к кол-ву сотрудников в этом периоде
\end{itemize}

Производительность труда измеряемую объемом произведенной продукции за ед. времени называют \textbf{выработкой} \\[0pt]

$\text{В} = \frac{\text{ВП}}{t} \ \text{или} \ \text{В} = \frac{\text{ВП}}{P_C}$
\\

Н\textsubscript{в} - выработка

ВП - объем произведенной продукции

t - затраты на раб.вр. на производство продукции

P\textsubscript{c} - средняя численность рабочих

Зависимость между трудоемкостью (\triangle T) и ростом выработки (\triangle \text{П}) определяется по формуле: \\[0pt]

\begin{math}
  \triangle \text{П} = [\frac{\triangle T}{100 * \triangle \text{П}}]
\end{math}


\begin{math}
  \triangle T = [\frac{\triangle \text{П}}{\triangle \text{П} * 100}]
\end{math}

Трудоемкость продукции представляет собой затраты живого труда на производство ед. продукции. Показатель трудоемкости (Q) устанавливает прямую зависимость между объемом продукции и трудовыми затратами \\[0pt]

$Q = \frac{t}{\text{ВП}}$
, где \\[0pt]

t - время, затраченное на производство на производство продукции (человеко-часы)

ВП - выпуск продукции

\begin{center}
Виды трудоемкости
\end{center}

\begin{itemize}
\item Технологическая (Q\textsubscript{тех}) - затраты труда основных рабочих-сдельщиков (t\textsubscript{сд}) и рабочих-повременьщиков (t\textsubscript{повр}) \\[0pt]

$Q_{\text{тех}} = t_{\text{сд}} + t_{\text{повр}}$
\end{itemize}


\begin{itemize}
\item Обслуживания производства (Q\textsubscript{обс}) - затраты труда основных и вспомогательных рабочих \\[0pt]

$Q_{\text{обс}} = t_{\text{вспом}} + t_{\text{всп}}$

\item Производственная (Q\textsubscript{пр}) - затраты труда всех рабочих \\[0pt]

$Q_{\text{пр}} = t_{\text{тех}} + t_{\text{обс}}$

\item Управления (Q\textsubscript{у}) - затраты труда служащих общезаводских служб предприятия \\[0pt]

$Q_{\text{у}} = t_{\text{слу}} + t_{\text{зав}}$
\end{itemize}


\begin{itemize}
\item Полная (Q\textsubscript{полн}) - затраты труда всех категорий рабочих \\[0pt]

$Q_{\text{полн}} = t_{\text{обс}} + t_{\text{тех}} + t_{\text{у}}$
\end{itemize}
\end{document}
