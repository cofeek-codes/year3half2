% Created 2024-06-03 Mon 00:57
% Intended LaTeX compiler: pdflatex
\documentclass[11pt]{article}
\usepackage[utf8]{inputenc}
\usepackage[T1]{fontenc}
\usepackage{graphicx}
\usepackage{longtable}
\usepackage{wrapfig}
\usepackage{rotating}
\usepackage[normalem]{ulem}
\usepackage{amsmath}
\usepackage{amssymb}
\usepackage{capt-of}
\usepackage{hyperref}
\usepackage{minted}
\usepackage[russian]{babel}
\author{Кормышев Егор ИСиП-301}
\date{02.03}
\title{Нормы труда}
\hypersetup{
 pdfauthor={Кормышев Егор ИСиП-301},
 pdftitle={Нормы труда},
 pdfkeywords={},
 pdfsubject={},
 pdfcreator={Emacs 29.3 (Org mode 9.6.15)}, 
 pdflang={Russian}}
\begin{document}

\maketitle
\tableofcontents

\textbf{Норма времени (Н\textsubscript{t})} - устанавливается в минутах или часах на одно изделие

$H_t = t_\text{п} + t_\text{оп} + t_\text{об} + t_\text{п}$ \\
$T_\text{шт} = t_\text{оп} + t_\text{об} * t_\text{п}$

, где

t\textsubscript{п} - время подготовки

t\textsubscript{оп} - основное время

t\textsubscript{об} - время обслуживания \\[0pt]

\textbf{Норма выроботки (H\textsubscript{в})} - кол-во продукции в ед. времени \\[0pt]

\begin{math}
H_\text{в} = \frac{1}{H_t}
\end{math}
\\[0pt]

\textbf{Норма обслуживания (H\textsubscript{обс})} - кол-во оборудования, которое должен обслужить 1 человек \\[0pt]

\textbf{Норма управляемости (H\textsubscript{у})} - оптимальное кол-во персонала \\[0pt]

\textbf{Норма численности (H\textsubscript{ч})} - устанавливается для некоторых категорий работников, если нет возможности рассчитать норму по другим способам
\end{document}
