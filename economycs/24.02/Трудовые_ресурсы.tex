% Created 2024-03-01 Fri 22:47
% Intended LaTeX compiler: pdflatex
\documentclass[11pt]{article}
\usepackage[utf8]{inputenc}
\usepackage[T1]{fontenc}
\usepackage{graphicx}
\usepackage{longtable}
\usepackage{wrapfig}
\usepackage{rotating}
\usepackage[normalem]{ulem}
\usepackage{amsmath}
\usepackage{amssymb}
\usepackage{capt-of}
\usepackage{hyperref}
\usepackage{minted}
\usepackage[russian]{babel}
\author{Кормышев Егор ИСиП-301}
\date{\today}
\title{Трудовые ресурсы}
\hypersetup{
 pdfauthor={Кормышев Егор ИСиП-301},
 pdftitle={Трудовые ресурсы},
 pdfkeywords={},
 pdfsubject={},
 pdfcreator={Emacs 28.2 (Org mode 9.5.5)}, 
 pdflang={Russian}}
\begin{document}

\maketitle
\tableofcontents



\section{Персонал организации, признаки и классификация, структура}
\label{sec:org80965f4}

Трудовые ресурсы (персонал, кадры) -  работники предприятия, выполняющие
управленческие, организационные и производственные функции. Уровень квалификации
работников определяется трудовой принадлежностью и выбранной профессии. Выбор
профессии зависит от приобретенной специальности, т.е. той суммы знаний конкретной
области деятельности, которую работник приобрел в учебном заведении.

\subsection{Структура трудовых ресурсов}
\label{sec:org3fa83bd}

Состав и количественные соотношения отдельных категорий и групп работников
предприятия характеризует структуру кадров.

Кадры предприятия (персонал, трудовые ресурсы) непосредственно связанные с
процессом производства продукции, т.е. занятые основной производственной
деятельностью, представляют промышленно – производственный персонал (ППП).

Работники торговли и общественного питания жилищного хозяйства, медицинских и
оздоровительных учреждений, учебных заведений и курсов, а также учреждений
дошкольного воспитания и культуры, состоящих на балансе предприятия относятся к
непромышленному персоналу предприятия (НПП).

Работники ППП подразделяются на две основные группы – рабочие и служащие. При
этом в группе служащих выделяются такие категории работающих, как руководители,
специалисты и собственно служащие. Рабочие обычно подразделяются на основных и
вспомогательных.
К
\emph{руководителям}
относятся лица, наделенные полномочиями принимать
управленческие решения и организовывать их выполнение. Они подразделяются на
линейных, возглавляющих относительно обособленные хозяйственные системы, и
функциональных, возглавляющих функциональные отделы или службы (главный
бухгалтер, главный инженер, лавный механик, главный технолог, главный экономист и
др.).
К
специалистам
относиться относятся работники, занятые инженерно –
техническими,
экономическими,
бухгалтерскими,
юридическими
аналогичными видами деятельности.
К собственно служащим относятся работники, осуществляющие подготовку и
оформление документации, учет и контроль, хозяйственное обслуживание и дело
производство (агенты, кассиры, контролеры, чертежники и др.)
Соотношение перечисленных категорий работников в общей их численности,
выраженное в процентах, называется структурой кадров.

\begin{enumerate}
\item Управление кадрами
\label{sec:org9289d13}

Представляет собой часть менеджмента, связанную с трудовыми ресурсами
предприятия. Его основными задачами являются удовлетворение потребности
предприятия в кадрах; обеспечение расстановки, профессионально – квалификационного
и должностного продвижения кадров; эффективное использование трудового потенциала
предприятия.
Обеспечение потребности в кадрах действующего предприятия предполагает не
только определение численности работников, но и текучести кадров, определение
дополнительной потребности или избытка кадров.
Под текучестью кадров понимается выраженная в процентах отношения числа
уволенных по собственному желанию, за прогулы и другие нарушения трудовой
дисциплины работников за определенный период времени к среднесписочной их
численности за тот же период времени. Она определяется по формуле:

\begin{equation}
k_{\text{тек}} = (P_{\text{уф}}/P_{\text{сп}})*100\%  
\end{equation}
\end{enumerate}
\end{document}