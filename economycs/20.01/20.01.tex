% Created 2024-06-02 Sun 15:39
% Intended LaTeX compiler: pdflatex
\documentclass[11pt]{article}
\usepackage[utf8]{inputenc}
\usepackage[T1]{fontenc}
\usepackage{graphicx}
\usepackage{longtable}
\usepackage{wrapfig}
\usepackage{rotating}
\usepackage[normalem]{ulem}
\usepackage{amsmath}
\usepackage{amssymb}
\usepackage{capt-of}
\usepackage{hyperref}
\usepackage{minted}
\usepackage[russian]{babel}
\author{Кормышев Егор}
\date{20 января}
\title{Лекция 2. Основные фонды}
\hypersetup{
 pdfauthor={Кормышев Егор},
 pdftitle={Лекция 2. Основные фонды},
 pdfkeywords={},
 pdfsubject={},
 pdfcreator={Emacs 29.3 (Org mode 9.6.15)}, 
 pdflang={Russian}}
\begin{document}

\maketitle
\tableofcontents

\textbf{Основные фонды} - часть имущества, которая используется в качестве средств труда длительное время, не изменяя свои свойства и состояние \\[0pt]

\textbf{Виды основных средств}:

\begin{itemize}
\item Производственные
\item Непроизводственные
\end{itemize}

\textbf{Группы основных средств}:

\begin{itemize}
\item Здания
\item Сооружения
\item Хоз.инвентарь
\item Инструменты
\end{itemize}
и т.д.

\section{Виды стоймостей основных фондов:}
\label{sec:org98cee46}

\begin{enumerate}
\item Первоначальная стоймость включает в себя затраты на приобретение единицы фонда с учетом затрат на транспортировку, монтаж и установку

$C_{\text{перв}} = Ц_{\text{пр}} * (1+a_{\text{трансп}} + \text{ам.иу}) * A_{\text{прин}}$
, где \\[0pt]
 a\textsubscript{\text{трансп}} - коэфицент, учитывающий транспортные затраты = 0,06 - не более 6\% от цены приобретения

\text{ам.иу} - коэфицент, учитывающий затраты на монтаж и устновку = 0,07 - не более 7\%

A\textsubscript{\text{прин}} - принятые единицы оборудования

\item Восстановительная стоймость включает в себя затраты на выпуск единицы фонда в период переоценки

\item Остаточная стоймость - часть основных фондов, которые не перенесены на готовую продукцию в виде амортизации

$C_{\text{ост}} = C_{\text{перв}} - \sum A * T$
, где \\[0pt]
\(\sum\) A - сумма амортизациионных отчислений \\[0pt]
T - период эксплуатации

\item Среднегодовая стоймость - стоймость - учитывается движение средств в ходе реализации и приобретения ОФ

\begin{math}
  \overline{C} = C_{\text{н.г}} + \frac{\text{Св.в} * t_1}{12} - \frac{С_{\text{Выб}} * t_2}{12}
\end{math}
, где

C\textsubscript{\text{н.г}} - стоймость ОФ на начало года \\[0pt]
C\textsubscript{\text{вв}} - стоймость введенных ОФ \\[0pt]
C\textsubscript{\text{выб}} - стоймость выбывших ОФ \\[0pt]
t - период эксплуатации фонда в течении года \\[0pt]
\end{enumerate}


\begin{enumerate}
\item Балансовая стоймость - стоймость, по которой объект учитывается в балане предприятия
\end{enumerate}

\begin{center}
\textbf{Стоймость ОФ на конец года}
\end{center}

$C_{\text{к}} = C_{\text{н.г}} + C_{\text{вв}} - C_{\text{выб}}$
, где \\[0pt]

C\textsubscript{\text{н.г}} - стоймость ОФ на начало года

C\textsubscript{\text{вв}} - стоймость введенных ОФ

C\textsubscript{\text{выб}} - стоймость выбывших ОФ
\end{document}
