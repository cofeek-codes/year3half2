% Created 2024-06-03 Mon 08:19
% Intended LaTeX compiler: pdflatex
\documentclass[11pt]{article}
\usepackage[utf8]{inputenc}
\usepackage[T1]{fontenc}
\usepackage{graphicx}
\usepackage{longtable}
\usepackage{wrapfig}
\usepackage{rotating}
\usepackage[normalem]{ulem}
\usepackage{amsmath}
\usepackage{amssymb}
\usepackage{capt-of}
\usepackage{hyperref}
\usepackage{minted}
\usepackage[russian]{babel}
\date{}
\title{}
\hypersetup{
 pdfauthor={},
 pdftitle={},
 pdfkeywords={},
 pdfsubject={},
 pdfcreator={Emacs 28.2 (Org mode 9.5.5)}, 
 pdflang={Russian}}
\begin{document}


Composer - пакетный менеджер зависимостей, предназначенный для упрощения загрузки и установки сторонних php библиотек в проект \\
PHP Unit - библиотека для проведения модульных тестов \\
Автоматизированное тестирование - тестирование с использование специального ПО \\
Альфа тестирование - приемочное тестирование на поздней стадии и включает имитацию реального использования \\
Баг-репорт- отчет об ошибках \\
Белый ящик - тестирование зная внутреннюю структуру программы или системы  (видя код) \\
Бета тестирование - интенсивное использование почти готовой версии продукта \\
Валидация - Определение соответствия разрабатываемого ПО ожиданиям и потребностям пользователя, требованиям системы \\
Валидация сайта - проверка соответствия программного кода ресурсам всем установленным и общепринятым нормам разработки, верстки и веб-дизайна \\
Верификация - процесс оценки системы или ее компонентов с целью определения того, удовлетворяют ли результаты текущего этапа разработки условиям сформированным в начале этого этапа \\
Гамма-тестирование - финальная стадия тестирования перед выпуском продукта \\
Графический интерфейс пользователя - разновидность интерфейса, обеспечивающая взаимодействие через графические элементы(меню, кнопки, значки и т.д.) \\
Доступность - система должна быть доступна авторизованным пользователям для чтения и записи данных \\
Жизненный цикл тестирования (ЖЦ STLC) - процесс тестирования, которое включает в себя определенную последовательность шагов, чтобы гарантировать достижение цели в области качества \\
Интеграционное тестирование - проверка взаимосвязей компонентов между собой и их интеграции с системой \\
Конфигурационное тестирование - тестирование во время которого проверяется работа ПО в различных конфигурациях системы \\
Конфиденциальность - посторонние пользователи не могут прочитать какие-либо данные \\
Критерии начала - набор условий для продолжения процесса с определенной задачей \\
Критерии окончания/выхода - набор условий для того чтобы процесс мог считаться завершенным \\
Модульное тестирование - проверка атомарных частей кода (методов, классов, функций, методов классов) \\
Нагрузка на сайт - Количество запросов, которое поступает на сервер веб-приложения в единицу времени \\
Негативное тестирование - тестирование системы на нештатное поведение \\
Объемное тестирование - вид тестирования, в котором производится проверка производительности веб-приложения при работе с большим объемом данных \\
Отчет по тестированию - содержит подробные ответы на вопросы: что тестировать? где тестировать? когда тестировать? как тестировать? \\
Позитивное тестирование - проверка работы системы на соответствие ее нормальному поведению согласно ТЗ \\
Покрытие кода - оценка покрытия исполняемого кода тестами, путем отслеживания непроверенных в процессе тестирования частей ПО \\
Покрытие требований - оценка покрытия тестами функциональных и нефункциональных требований к продукту, путем построения матриц трассировки \\
Пользовательский интерфейс - совокупность средств и методов, при помощи которых пользователь взаимодействует с различными машинами, устройствами и аппаратурой \\
Приемочное тестирование (E2E, сквозное) - тестирование, во время которого происходит валидация требований \\
Распознавание образа - подход использующий механизмы реализации специальных возможностей (accessibility) и особенности реализации некоторых UI-фреймворков \\
Регрессионное тестирование - тестирование которое направленно на проверку изменений, сделанных в приложении для подтверждения того что существующая функциональность работает как прежде \\
Санитарное тестирование или проверка согласованности/исправности - тестирование которое доказывает то, что конкретная функция работает согласно заявленным спецификациям требований \\
Серый ящик - тестирование частично зная внутреннюю структуру программы или системы (частично видя код) \\
Системное тестирование - проверка взаимодействия тестируемого ПО с системой по функциональным и нефункциональным требованиям с максимально приближенным окружением, которое у конечного пользователя \\
Стрессовое тестирование - вид тестирования, в ходе которого производится проверка производительности веб-приложения при экстремально высоких нагрузках \\
Тест-дизайн - этап процесса тестирования ПО на котором проектируются и создаются тест-кейсы, в соответствии с определенными ранее критериями качества и целями тестирования \\
Тестирование безопасности - тестирование в котором необходимо найти и устранить уязвимости в веб-приложении до того, как ими воспользуются злоумышленники \\
Тестирование нагрузки сайта - процесс проверки производительности с помощью симуляции повышенного количества запросов \\
Тестирование на отказ и восстановление - тестирование во время которого проверяется способность на отказ и успешность восстановления после сбоев \\
Тестирование пользовательского интерфейса - тестирование, во время которого проверяется взаимодействие человека и ПО \\
Тестирование ПО - Процесс анализа программного средства и сопутствующей документации с целью выявления дефектов и повышения качества продукта \\
Тестирование производительности - тестирование, задачей которого является определение масштабирования приложения под нагрузкой \\
Тестирование сборки - тестирование которое проверяет соответствие выпущенной версии критериям качества \\
Тестирование стабильности/надежности - вид тестирования, в котором производится проверка производительности веб-приложения при продолжительной работе \\
Тестирование установки - тестирование, которое направлено на проверку успешной установки и настройки а также на обновление и удаление ПО \\
Тест-кейс - документ, который описывает последовательность шагов, условий и параметров, необходимых для проверки объекта тестирования \\
Тестовое покрытие - на базе анализа потока управления - оценка покрытием основанное на определении путей выполнения кода программного модуля и создания выполняемых тест-кейсов для покрытия этих путей \\
Тест-план - вид тестовой документации, который отвечает на вопросы: что тестировать? где тестировать? когда тестировать? как тестировать? \\
Тест-сьют - набор тест-кейсов, объединенный одним модулем, функциональностью, приоритетом и т.д. \\
Требование - описание того, какие функции и соблюдение условий должно выполнять \\
Функциональное тестирование - тестирование, которое основывается на функциях и особенностях, а также на взаимодействии с другими системами \\
Целостность - Неавторизованные пользователи не могут писать данные \\
Чек-лист - документ содержащий список проверок необходимых в рамках тестирования \\
Черный ящик - тестирование без доступа к внутренней системе (не видя код) \\
\end{document}
